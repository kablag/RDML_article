\documentclass{bioinfo}
\copyrightyear{2014}
\pubyear{2014}

\begin{document}
\firstpage{1}

\title[RDML]{RDML: an R Package for RDML Format Data Import}
\author[Sample \textit{et~al}]{Corresponding Author\,$^{1,*}$, Co-Author\,$^{2}$ and Co-Author\,$^2$\footnote{to whom correspondence should be addressed}}
\address{$^{1}$Department of XXXXXXX, Address XXXX etc.\\
$^{2}$Department of XXXXXXXX, Address XXXX etc.}

\history{Received on XXXXX; revised on XXXXX; accepted on XXXXX}

\editor{Associate Editor: XXXXXXX}

\maketitle

\begin{abstract}

\section{Motivation:}
One of the recommendated elements in the Minimum Information for Publication of Quantitative 
Real-Time PCR Experiments (MIQE) guidelines is RDML file format. Despite the fact that the ability to export data in this format is implemented in software products of major PCR instruments, there is no tool allowing to work with the RDML format in R.
\section{Results:}
We developed R package called RDML, which (i) reads fluorescence data obtained during qPCR and melting experiments and (ii) gets information about samples type, targets and concentration from RDML files generated by varius PCR instruments; (iii) transforms fluorescence data to the appropriate format of the \textit{qpcR} and \textit{chiPCR} packages; (iv) generates human readable summary about all samples in experiment.  
\section{Availability:}
The R package RDML is implemented using R, and can be used in Windows, Mac, or Linux environments. It is available at CRAN and the latest version at  https://github.com/kablag/RDML/.
\section{Contact:} \href{k.blag@yandex.ru}{k.blag@yandex.ru}
\end{abstract}

\section{Introduction}

Quantitative real-time PCR (qPCR) based on fluorescence detection is one the most popular methods in molecular biology, genetic, life science, agriculture and medicine disciplines. This method is fairly simple and at the same time provides accurate, reliable and reproducible information\cite{kubista_real-time_2006}. Second actively growing recently method with a fluorescence data using is a high-resolution melting \cite{reed_high-resolution_2007}\cite{wittwer_high-resolution_2009}\\
There are about twenty available real-time PCR cycler manufacturers, which in turn offers one or more models. Such a variety of models and manufacturers of both advantage and disadvantage. Users can select the appropriate models to them by number of samples at one run, the number of dyes, the method of work, the price and other parameters of the model. However, the majority of devices, even within a single manufacturer model range, using different methods of data processing and incompatible data storage formats. Applying different methods of data processing causes the PCR results obtained on different devices can not reliably be compared with results obtained by other devices. On the other hand, various storage formats do not allow to process the raw data from one instrument to another instrument software, and, more importantly, with the use of independent reference methods.\\
At this point, there is a document designed to allow comparison the results of qPCR and HRM analysis performed on different instruments -- MIQE guidelines \cite{bustin_miqe_2009}. These guidelines suggest using the Real-time PCR Data Markup Language (RDML) as the main interchange format for PCR and HRM data. RDML data standard contains sufficient information to understand the experimental setup, re-analyse the data and interpret the results \cite{lefever_rdml:_2009}. But despite the presence of RDML format there is a lack in the software capable of working with data in this format. Basically it is programs from manufacturers of PCR devices or commercial programs such as \textit{qbase+} \cite{_rdml_????}. On the other hand, there are a number of open source processing solutions for PCR and HRM data, for example, implemented in R packages \textit{qpcR}\cite{ritz_qpcr:_2008} and \textit{chipPCR}.\\
In this article we describe our R package \textit{RDML}which allows to import qPCR data from RDML v1.1 format files and transform it to the human readable format appropriate to the \textit{qpcR} and \textit{chipPCR} packages. \textit{RDML} package is open-source, free to use, platform-independent, and thus can be modified for specific tasks or other R packages, even integrated to work-flows with other programming languages.

\section{Implementation}
\section{Results and discussion}

%\bibliographystyle{natbib}
%\bibliographystyle{achemnat}
%\bibliographystyle{plainnat}
%\bibliographystyle{abbrv}
%\bibliographystyle{bioinformatics}
%
\bibliographystyle{plain}
%
\bibliography{RDML}

\end{document}
