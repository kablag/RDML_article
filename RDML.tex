% !TeX spellcheck = en_US
\documentclass{bioinfo}

\usepackage{hyperref} \usepackage{tikz} \usetikzlibrary{graphs}

\let\proglang=\textsf

\copyrightyear{2015} \pubyear{2015}

\begin{document} \firstpage{1}
	
\title[RDML]{RDML: an \textbf{R} Package for Working with RDML Format Data}
\author[Blagodatskikh \textit{et~al}]{Konstantin Blagodatskikh\,$^{1}$\footnote{to whom correspondence should be addressed}, Micha\l{} Burdukiewicz\,$^{2}$, Stefan R\"{o}diger\,$^{3}$ 
	and Andrej-Nikolai Spiess\,$^{4}$}

\address{$^{1}$All-Russia Research Institute of Agricultural Biotechnology, Moscow, Russia\\ 
	 $^{2}$Department of Genomics, Faculty of Biotechnology, University of Wroc\l{}aw, Wroc\l{}aw, Poland\\
	 $^{3}$Faculty of Natural Sciences, Brandenburg University of Technology Cottbus--Senftenberg, Germany\\ 
	 $^{4}$University Medical Center Hamburg-Eppendorf, Hamburg, Germany
	 }
	
\history{Received on XXXXX; revised on XXXXX; accepted on XXXXX}

\editor{Associate Editor: XXXXXXX}
	
\maketitle

\begin{abstract}
		
\section{Motivation:} Reproducible research is an essential 
element for good scientific conduct. Part of this is are open data exchange 
formats. The  Real-time PCR Data Markup Language (RDML) file format is a 
recommended element of the Minimum Information for Publication of Quantitative 
Real-Time PCR Experiments (MIQE) guidelines. Preserving the set-up and output of 
experiments in well-defined scheme, RDML is a universal data storage systems. 
Despite the fact that software products of qPCR instruments export RDML data, 
there is no tools designed for manipulating RDML data in sophisticated 
environment for statistical bioinformatics.
		
\section{Results:} We developed the cross-platform open source \textit{RDML} 
package for the  \textbf{R} statistical computing language. \textit{RDML} is 
compliment to RDML~$\geq$~v.~1.1 and provides functionality to (i) import RDML 
data; (ii) to extracts sample information (e.g., targets, concentration); (iii) 
to transform data to various formats of the \textbf{R} environment; (iv) to 
generate human readable experiment summaries; and (v) to create RDML files from 
user data. In addition, offers \textit{RDML} a graphical user interface for 
selected functions. We show that \textit{RDML} enables a seamless analysis with 
with other \textbf{R} packages.

\section{Availability:} url{https://cran.r-project.org/web/packages/RDML}. Source code: 
url{https://github.com/kablag/RDML}. \section{Contact:} 
\href{k.blag@yandex.ru}{k.blag@yandex.ru} \end{abstract}


\section{Introduction}
  Real-time quantitative PCR (qPCR) is among the most used methods in molecular 
biology, diagnostics, food industry and other areas. qPCR is well accepted since 
it is simple, accurate, reliable and reproducible \cite{pabinger_2014}. The 
Minimum Information for Publication of Quantitative Real-Time PCR Experiments
were established to facilitate the comparison of experimental results from qPCR 
and digital PCR (dPCR) \cite{huggett_2013}. There are at least twenty real-time 
quantitative PCR cycler manufacturers. Such variety creates a broad selection to 
address the specific market demands. The systems differ in details such as the 
number of processable samples, the number of detection channels, the sample 
arrangement (e.g., plate, carousel). In addition to this, the devices use different 
data processing methods and incompatible data storage formats (e.g.,  binary formats). 
The binary data format and system specific formats result in most cases a vendor 
lock-in. Therefore, the diversification comes at a cost at the level of data 
exchange. Discrepancies in data processing hinder comparison of the PCR results 
obtained on different systems. Disparate storage formats prohibit the processing 
of raw data from one instrument in the another system. Data processing is a 
central challenge in qPCR data analysis \cite{roediger2015r, 
spiess_impact_2014}.

The MIQE guidelines suggest the Real-time PCR Data Markup Language (RDML) as 
qPCR interchange data format \cite{rdml-ninja_2015}. RDML is a vendor 
independent and freely available file format, based on eXtensible Markup 
Language (XML), which is commonly used in bioinformatics \cite{achard_xml_2001}, 
to pass meta data and value between applications. Storage and exchange of qPCR 
data is bidirectional. The RDML data standard contains the primary raw data 
acquired by the machine, meta-information to understand the experimental setup 
(e.g., sample annotation, qPCR protocol, probe and primer sequences) and 
information to re-analyse the data \cite{lefever_rdml_2009}. The RDML format is 
supported by different vendors (see Supplement) and third-party software (e.g., 
\textit{Primer3Plus} \cite{untergasser_2007}, \textit{QPCR}, \textit{LinRegPCR}, 
\textit{qBase+}) \cite{pabinger_2014, rdml-ninja_2015}. Despite the openness of 
the RDML format, there is no open source software capable of reading and 
processing this data. Users have to choose between programs from manufacturers 
of PCR devices (which can use only their own generated files) or commercial 
software such as \textit{qbase+} \cite{pabinger_2014, rdml-ninja_2015}. 
Recently, Ruijter~\textit{et~al.} published the open-source desktop software 
\textit{RDML-Ninja}, which can visualize, edit and validate RDML files 
\cite{rdml-ninja_2015}. However, \textit{RDML-Ninja} cannot analyze the qPCR 
data by sophisticated analysis pipelines. Additionally, there is a need for a 
\textit{de novo} creation of RDML data.  It is especially of importance for 
systems, that do not support the RDML format. 
This includes commercial systems and experimental systems (e.g.,  point-of-care 
devices). 

The \textit{RDML} package for the statistical computing language \textbf{R} 
closes this gap. \textbf{R} runs on virtually any platform and provides a 
comprehensive set of tools for reproducible research preserving objects created 
during the analysis \cite{roediger2015r,roediger2015chippcr}. This includes 
standalone desktops or server structures \cite{roediger2015r}. There is a number 
of open source processing solutions for qPCR, dPCR and melting curve analysis 
written in \textbf{R} \cite{pabinger_2014, ritz_qpcr_2008, roediger_RJ_2013, 
roediger2015chippcr}. Although the \textbf{R} environment extended by dedicated 
packages provides wide array of tools, it was not possible to seamlessly process 
RDML files.

The common approaches for data management in \textbf{R} are native formats such 
as and \textbf{R} workspaces or objects \cite{roediger_rkward_2012}. However, 
the external applications of these methods is limited regarding the RDML format, 
a specifically tailored variant of XML. The principle of reproducible research 
is not retained without a standard method of data import as the \textit{RDML} 
package. Our software enables usage of the qPCR-related \textbf{R} tools as 
recently reviewed in \cite{pabinger_2014}, while working on the RDML data 
derived straight from the PCR system. Concluding, the \textit{RDML} package may 
serve as foundation for other \textbf{R} package hosted at Bioconductor 
\cite{gentleman_2004} and CRAN as it was shown recently in \cite{roediger2015r}.

In this article we describe the \textbf{R} package \textit{RDML}, which 
allows to exchange RDML files ($\geq$~v.~1.1) and transform it to the human 
readable format. Until version \textit{RDML} package 0.8-3, only RDML~v.~1.1 and 
file import was supported. Here we present a considerably maturated version of 
the package. \textit{RDML} package is open-source, free to use, 
platform-independent, and thus can be modified for specific tasks or other 
\textbf{R} packages, or even integrated to work-flows with other programming 
languages. Despite many qPCR instruments and software solutions are able 
to work with the RDML format, there is a need make RDML available for the increasingly 
used \textbf{R} environment.

\section{Implementation}
	
The \textit{RDML} package is available at CRAN 
(\url{http://cran.r-project.org/web/packages/RDML/index.html}) and the 
development version is hosted at GitHub (\url{https://github.com/kablag/RDML}) 
with facilities for bugs reports and request features. \textit{RDML} provides 
\emph{R6} classes, which corresponds to RDML~v.~1.2 format types. \textbf{R} 
is a language with dynamic typing. This is helpful while scripting but can lead 
to problematic debugging of more complicated workflows. \emph{R6} classes 
provide type-safe interfaces to set data without access to inner structure of 
objects thus all imputed data can be validated. This option is very useful when 
creating packages to work as intermediate level for other packages (e.g., such 
approach does not allow set \emph{character} in place of \emph{integer} 
Supplementary Section S3). Furthermore, the inheritance of \emph{R6} objects 
unifies the structure of the package and streamlines extending its capabilities 
(the whole package is written around a single base class). The main interaction 
is via the command line interface (CLI). However, we designed a graphical user 
interface based on the \textit{shiny} technology as described in 
\cite{roediger2015chippcr}. The RDML file format, coordinated by the RDML 
consortium (www.rdml.org/), is under continuous development. The \textit{RDML} 
package follows the changes and supports the current version RDML~v.~1.2. 
Central functionality of the \textit{RDML} package encompass the read-in of RDML 
data file and summary generation for RDML objects. The capabilities of packages 
range from advanced statistical analysis to simple yet crucial extraction of the 
fluorescence data. The public methods of the main class called RDML can be used 
to access and process the RDML data. These methods include:

\begin{itemize} 
  \item \textbf{\$new()} -- creates new RDML object. Empty or from specified RDML 
  file (Supplementary Section S1); \item \textbf{\$AsDendrogram()} -- represents 
  structure of RDML object as dendrogram to observe it. (Supplementary Section 
  S1);
  \item \textbf{\$AsTable()} -- represents data contained in RDML object (except 
  fluorescent data) as \textit{data.frame} (Supplementary Section S4);
  \item \textbf{\$GetFData()} -- gets fluorescent data (Supplementary Section S5);
  \item \textbf{\$SetFData()} -- sets fluorescent data to RDML object (Supplementary Section S6); 
  \item \textbf{\$AsXML()} -- saves RDML object as RDML~v.~1.2 file (Supplementary Section S7).
\end{itemize}
	
	Beside working with existing RDML files, the package can create new RDML 
objects from the data obtained from systems that do not support RDML format. To 
create eligible RDML object, one has to provide fluorescence data and minimal 
description of experiment (Supplementary Section S10). The another unique feature of our 
package is the ability to merge several RDML files (experiments) into one with 
\textbf{MergeRDMLs()} function and process them like unit (Supplementary Section 
S7). For example, you can combine results of two runs that have sample from one 
experiment or add calibration samples from one run to another.

Data sets are an essential element of \textbf{R} packages for reproducible 
research \cite{roediger2015r}. As of \textit{RDML} v.~0.9-1 we provide data sets 
from the Bio-Rad CFX 96, Roche LightCycler 96 and Applied Biosystems StepOne.

\section{Discussions and conclusions}
	
	The \textit{RDML} package for \textbf{R} imports and manipulates data 
from RDML~v.~1.1 and v.~1.2 format files. It can also create RDML~v.~1.2 files 
from data provided by an user. This package can be used as a part of the qPCR 
processing workflows or for preliminary summaries of experiments. Due to the 
openness of the format, we anticipate that other scientific disciplines might 
benefit from this implementation.

Researchers can no longer collect data in a single laboratory and are often are 
coerced to work with many data sets. Adaptable data management - aka. adaptive 
informatics - is relevant in cases where data from omics approaches and 
different assays (e.g., flow cytometry, digital PCR, NGS) need to be merged 
\cite{baker_quantitative_2012}. Due to the fact that RDML is based on XML it is 
possible to converge the file with other formats such as HDF 
\cite{millard_adaptive_2011}. Since R offers also HDF5 interface 
\cite{Fischer_HDF5}, the \textit{RDML} package facilitates not only data storage 
and analysis, but also the higher level of the experimental data management.

\section{Acknowledgements}
Grateful thanks belong to the \textbf{R} community and the RDML consortium.
	
\paragraph{Funding\textcolon} This work was funded by the Russian 
Ministry of Education and Science (project No. RFMEFI62114X0003), with 
usage of scientific equipment of the Center for collective use ’Biotechnology’ at 
All-Russia Research Institute of Agricultural Biotechnology and the Federal 
Ministry of Education and Research (BMBF) InnoProfile--Transfer--Projekt 03 IPT 
611X.
	
\paragraph{Conflict of Interest\textcolon} none declared.

%\bibliographystyle{natbib}
%\bibliographystyle{achemnat}
%\bibliographystyle{plainnat}
%\bibliographystyle{abbrv}
%\bibliographystyle{bioinformatics}
%
\bibliographystyle{plain}
%
\bibliography{RDML}

\end{document}
